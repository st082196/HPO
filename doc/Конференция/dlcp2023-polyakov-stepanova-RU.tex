\documentclass[reprint,
superscriptaddress,
amsmath,amssymb,aps,showkeys,showpacs,
twoside,final,secnumarabic,%raggedbottom,
nofootinbib]{revtex4-2}

%======Размеры бумаги и~текста=======
\usepackage[paperwidth=205mm,paperheight=290mm,top=17mm,bottom=25mm,
%,16mm,% inner=18mm,% outer=15mm,
inner=17mm,outer=17mm,
twoside]{geometry}

%================Пакеты=====================
\usepackage{cmap} % Улучшенный поиск %русских слов в~полученном pdf-файле
\defaulthyphenchar=127 % Если стоит до fontenc, то переносы не впишутся в~выделяемый текст при копировании его в~буфер обмена
\usepackage[T1,T2A]{fontenc}
\usepackage[utf8]{inputenc}
%\usepackage[cp1251]{inputenc}
\usepackage[russian,english]{babel}
\usepackage{color}
%\usepackage{physics}
\usepackage{graphicx}% Include figure files
\usepackage{dcolumn}% Align table columns on decimal point
\usepackage{bm} % bold math
\usepackage[unicode=true,colorlinks=true,linkcolor=magenta, urlcolor=blue, citecolor = blue,breaklinks]{hyperref}
% add hypertext capabilities
\usepackage{multirow}
\usepackage{url}
\def\UrlBreaks{\do\/\do-}
\usepackage{breakurl}
\newcommand\sbullet[1][.5]{\mathbin{\vcenter{\hbox{\scalebox{#1}{$\bullet$}}}}}
%========Для работы на компе==================
%\usepackage{../latex/breakurl}
%\graphicspath{{../figures/}}
\DeclareGraphicsExtensions{.eps}

%===========Счетчики================
\newcount\issue
\newcount\Vol
\newcount\numb
% ======Колонтитулы==============
\headheight=1.5cm
\usepackage{fancyhdr} %this packages %provides fancy up and bottom of page
\pagestyle{fancy}
\fancyhead{}\fancyfoot{}
%\footheight=0.5cm
\fancyfoot[LO]{}
\fancyfoot[CO]{\small{\numb--\thepage}}
\fancyfoot[RO]{}
\fancyfoot[LE]{}
\fancyfoot[CE]{\small{\numb--\thepage}}
\fancyfoot[RE]{}
\fancyhead[CO]{\normalsize\textrm{Moscow University Physics Bulletin \Vol(\the\issue)},~\numb~(\the\year)}
% {текст-центр-нечетные}

\fancyhead[CE]{\normalsize\selectlanguage{english}{Modern Machine Learning Methods}}
% {текст-центр-четные}

%=======Ссылка на почту=======
\newcommand\mailto[1]{\href{mailto:#1}{#1}}
%============================
\renewcommand\thesection{\arabic{section}}
\renewcommand\thetable{\arabic{table}}

%============================
\year2023 \issue7
%\sheets999
\def\Vol{\textbf{78}}
\def\numb{x}
\setcounter{page}{1}
%============================

\begin{document}

%====== Начало шапки статьи  ============
\title{MODERN MACHINE LEARNING METHODS\\[20pt]
Выбор гиперпараметров нейронной сети для решения \\
задач большой размерности в случае уравнения Гельмгольца}

\def\address{Кафедра вычислительной физики, Санкт-Петербургский государственный университет,\\
	Россия, 199034, Санкт-Петербург, ул.Ульяновская, 1.}

\author{\firstname{Д.Н.}~\surname{Поляков}}
\email[E-mail: ]{st082196@student.spbu.ru}
\affiliation{\address}
\author{\firstname{М.М.}~\surname{Степанова}}
\email[E-mail: ]{m.stepanova@spbu.ru}
\affiliation{\address}

\received{xx.xx.2023}
\revised{xx.xx.2023}
\accepted{xx.xx.2023}

\begin{abstract}
В работе исследуется эффективность различных методов подбора гиперпараметров для Physics-Informed Neural Network (PINN) на примере решения многомерного уравнения Гельмгольца. Нейронная сеть была построена на фреймворке PyTorch без использования специальных библиотек для PINN-сетей. Исследовано влияние гиперпараметров на производительность нейросети и проведена автоматическая оптимизация со сравнением популярных алгоритмов поиска и планировщиков обучения.

В качестве инструмента для оптимизации гиперпараметров (HPO) был выбран фреймворк Ray Tune с открытым исходным кодом, предоставляющий единый интерфейс для работы с множеством HPO-пакетов. Рассмотрены алгоритмы случайного поиска, байесовского поиска, основанного на дереве парзеновских оценок (TPE) в реализациях hyperopt и hpbandster, и алгоритм ранней остановки (ASHA). Использование алгоритма ранней остановки позволяет существенно быстрее получить лучшую конфигурацию гиперпараметров при любой размерности задачи.
\end{abstract}

%\pacs{Suggested PACS}\par
\keywords{HPO, PINN, PyTorch, Ray Tune, уравнение Гельмгольца\\[5pt]}
%DOI:  

\maketitle
\thispagestyle{fancy}

\end{document}